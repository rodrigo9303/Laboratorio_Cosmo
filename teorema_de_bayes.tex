\documentclass{report}
\usepackage[spanish,activeacute]{babel}
\usepackage[utf8]{inputenc}
\usepackage{multicol}
\usepackage{amsmath}
\usepackage{graphicx}
\begin{document}
	TEOREMA DE BAYES\\
	
	Recordemos que la probabilidad condicional se define como
	\begin{eqnarray*}
		&& P(A|B)=\dfrac{P(A\cap B)}{P(B)}=\dfrac{P(A,B)}{P(B)},\\ && por\enspace lo\enspace que \\ && P(A,B)=P(A|B)P(B).
	\end{eqnarray*}
	Entonces si se toma que $ P(A,B)=P(B,A) $, se tendr\'a 
	\begin{eqnarray*}
	    P(A,B)=P(B,A)\\ && P(A|B)P(B)=P(B|A)P(A)
	\end{eqnarray*}
	Finalmente 
	\begin{equation*}
	    P(A|B)=\dfrac{P(B|A)P(A)}{P(B)}.
	\end{equation*}
	Por lo que el teorema queda derivado.
\end{document}